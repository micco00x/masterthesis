\chapter{Variable Height CoM IS-MPC}
LIP, variable height CoM. MPC (BHuman implementation). Cite
\cite{SYROCO18}.

\section{LIP}
Where it comes from, proofs and prev. works.

3D CoM dynamics.

Moment balance around ZMP for the $x$ coordinate:
\begin{equation}
  (z_c - z_z) \ddot{x}_c = (\ddot{z}_c + g) (x_c + x_z)
\end{equation}
where $g$ is the gravity acceleration. Identical for $y$. Dynamics of $x$, $y$
coordinates of the CoM:
\begin{align}
  \ddot{x}_c &= \frac{\ddot{z}_c + g}{z_c - z_z} (x_c - x_z) \\
  \ddot{y}_c &= \frac{\ddot{z}_c + g}{z_c - z_z} (y_c - y_z)
\end{align}
Force balance along $z$ axis leads to structurally different dynamics for the 
corresponding coordinate:
\begin{equation}
  \ddot{z}_c = \frac{f_z}{m} - g
\end{equation}
where $f_z$ denotes the z-component of the ground reaction force (GRF), acting 
as an external input, and $m$ is the total mass of the humanoid.
SEE KAJITA HERE.

Condition for maintaining balance is that CoP is internal to SP. Since CoP, CoM,
ZMP are colinear, the condition is equivalent to the ZMP being internal to the 
polyhedral cone having CoM as vertex and SP as cross-section (Fig. 2 SYROCO18).

3D Linear Model (const height CoM) SECTION:

MAYBE MERGE THIS SECTION WITH THE PREV USING THEORY FROM KAJITA?

3D motion model of prev. section nonlinear, difficult to use for gait
generation. Usual way to make the model linear is to assume ground fully 
horizontal and CoM over the ground const, as a consequence $z_z = 0$, hence 
ZMP coincident with CoP. \textbf{LIP}:
\begin{align}
  \ddot{x}_c &= \omega_0^2 (x_c - x_z) \\
  \ddot{y}_c &= \omega_0^2 (y_c - y_z)
\end{align}
where $\omega_0^2 = g/\bar{z}_c$. 2D LINEAR MODEL NOT APPROPRIATE FOR GAIT
GENERATION OVER UNEVEN TERRAIN.

\section{3D Linear Model (variable height CoM)}

However, requiring CoM move at const height is not the only way to make the 
system linear. A more general way is to constraint its vertical motion so that:
\begin{equation}
  \frac{\ddot{z}_c + g}{z_c - z_z} = \omega^2
\end{equation}
with $\omega$ arbitrary const. LIP model can be seen as a particular case in
which $\omega^2 = \omega_0^2$. Using the above, eqs. CoM dynamics become:
\begin{align}
  \ddot{x}_c &= \omega^2 (x_c - x_z) \\
  \ddot{y}_c &= \omega^2 (y_c - y_z)
\end{align}
whereas the dynamics of $z_c$ is directly derived from the constraint itself:
\begin{equation}
  \ddot{z}_c = \omega^2 (z_c - z_z) - g
\end{equation}

The above dynamic equations are linear and have a clear LIP structure with 
ZMP coordinates $(x_z, y_z, z_z)$ acting as control inputs. THIS 3D MODEL 
ALLOWS VERTICAL MOTION OF THE COM AND THEREFORE CAN BE USED FOR GAIT GENERATION 
ON UNEVEN TERRAIN, IN CONJUNCTION WITH THE BALANCE CONDITION ILLUSTRATED IN 
FIG. 2 (SYROCO18).

DOUBLE CHECK NOTES ON SYROCO18 SEC. 2.2 PAGE 395.

\section{MPC Formulation + Algorithm}
Proofs as in report.pdf in order to easily introduce QP problem.

Assuming footstep sequence in given in advance: MPC scheme based on 3D model 
introduce above. Eqs. include an unstable subsystem, taken care by adding
stability constraint. Extends IS-MPC Scianca
\cite{DBLP:conf/humanoids/SciancaCSLO16}.

3.1. Motion Model

Dynamic extension: choose control variable as ZMP velocity $\dot{x}_z$ instead
of ZMP itself. On the x axis:
\begin{equation}
  \begin{pmatrix}
    \dot{x}_c \\
    \ddot{x}_c \\
    \dot{x}_z
  \end{pmatrix}
  =
  \begin{pmatrix}
    0 & 1 & 0 \\
    \omega^2 & 0 & -\omega^2 \\
    0 & 0 & 0
  \end{pmatrix}
  \begin{pmatrix}
    x_c \\
    \dot{x}_c \\
    x_z
  \end{pmatrix}
  +
  \begin{pmatrix}
    0 \\
    0 \\
    1
  \end{pmatrix}
  \dot{x}_z
\end{equation}
Dynamics are the same along three axes with additive term along z axis.

Using piecewise-constant control inputs over sampling intervals of duration 
$\delta$, with a prediction horizon $T_h = N \dot \delta$. Current time instant
$t_k$, successive instants within prediction horizon $t_k+i, i = 1, \dots, N$.
Similar notation for other variables. At a generic isntant $t_j$:
\begin{equation}
  \dot{x}_z(t) = \dot{x}_z^j, t \in [t_j, t_{j+1})
\end{equation}
so that the ZMP x-position in the time interval $[t_j, t_{j+1}]$ is:
\begin{equation}
  x_z(t) = x_z^j + (t - t_j) \dot{x}_z^j, t \in [t_j, t_{j+1}]
\end{equation}

3.2. ZMP constraints

2D case first. ZMP inside the SP. $(x_f^j, y_f^j, \theta_f^j)$ pose of the 
generic footstep within assigned sequence. Fixed-shape moving ZMP constraint to
enforce balance. Admissible region for ZMP at $t_{k+i}$ is centered in 
$(x_f^{k+i}, y_f^{k+i})$ and has orientation $\theta_f^{k+i}$.

SS: $(x_f^{k+i}, y_f^{k+i}, \theta_f^{k+i})$ coincide with the pose of the 
support foot, hence, $(x_f^j, y_f^j, \theta_f^j)$.

DS: $(x_f^{k+i}, y_f^{k+i}, \theta_f^{k+i})$ gradually slide from the position 
and orientation of the previous support foot to those of the next. [NOT 
STRICTLY NECESSARY, DOUBLE CHECK IN IMPLEMENTATION AS WELL].

ZMP constraint in 2D:
\begin{equation}
  \label{eq:zmp-constraint-2d}
  -\frac{1}{2}
  \begin{pmatrix}
    d_x^\text{z} \\
    d_y^\text{z}
  \end{pmatrix}
  \le
  R_{k+i}^T
  \begin{pmatrix}
    x_z^{k+i} - x_f^{k+i} \\
    y_z^{k+i} - y_f^{k+i}
  \end{pmatrix}
  \le
  \frac{1}{2}
  \begin{pmatrix}
    d_x^\text{z} \\
    d_y^\text{z}
  \end{pmatrix}
\end{equation}
where $d_x^\text{z}$ and $d_y^\text{z}$ are the dimensions of the rectangular
constraint region and $R_{k+i}^T$ is the rotation matrix associated with
$\theta_f^{k+i}$. Note that $(x_z^{k+i}, y_z^{k+i})$ is the predicted position
of the ZMP, which can be expressed as a linear combination of the control
variables:
\begin{equation}
  \label{eq:piecewise-linear-zmp-trajectory}
  x_z^{k+i} = x_z^k + \delta \sum_{i=0}^{i-1} \dot{x}_z^{k+j}
\end{equation}
Eq. \eqref{eq:zmp-constraint-2d} must be imposed for $i = 1, \dots, N$.

ZMP constraint in 3D:
ZMP is allowed to leave the ground in order to generate vertical CoM motions,
balance condition now requires ZMP to remain inside polyhedral cone defined 
by SP and CoM. When ZMP is allowed to move vertically, the cone defines a 
nonlinear constraint. In order to remove nonlinearity, box constraint:
\begin{equation}
  \label{eq:zmp-constraint-3d}
  -\frac{1}{2}
  \begin{pmatrix}
    \tilde{d}_x^\text{z} \\
    \tilde{d}_y^\text{z} \\
    d_z^\text{z}
  \end{pmatrix}
  \le
  R_{k+i}^T
  \begin{pmatrix}
    x_z^{k+i} - x_f^{k+i} \\
    y_z^{k+i} - y_f^{k+i} \\
    z_z^{k+i} - y_f^{k+i}
  \end{pmatrix}
  \le
  \frac{1}{2}
  \begin{pmatrix}
    \tilde{d}_x^\text{z} \\
    \tilde{d}_y^\text{z} \\
    d_z^\text{z}
  \end{pmatrix}
\end{equation}
where $d_z^\text{z}$ defines the maximum allowed vertical ZMP displacement
w.r.t. the horizontal patch. To guarantee that the box is consteind in the cone:
\begin{equation}
  \tilde{d}_x^\text{z} = d_x^\text{z} \left( 1 -
      \frac{d_z^\text{z}}{2z_c^{\min}} \right)
      - \frac{d_z^\text{z}}{z_c^{\min}}\Delta x_c
\end{equation}
where $\Delta x_c$ max expected displacement of CoM w.r.t. center of support 
foot and $z_c^{\min}$ is the minimum expected value for CoM height.
Analogous for $\tilde{d}_y^\text{z}$.

Similarly to 2D case, box constraint kept fixed during SS. During DS box 
slide linearly from its position around the previous support foot to its 
position around the next support foot, thus, always remaining within the 
polyhedral cone which defines the ZMP balance constraint (Fig. 5 SYROCO18).

3.3. Stability constraint

Motion model unstable, generic solution diverges making components of CoM 
unbounded w.r.t. ZMP position, generated gait infeasible.
In \cite{DBLP:conf/humanoids/SciancaCSLO16} it was shown (TODO: report proof 
here) that if the initial condition $(x_c^k, \dot{x}_c^k)$ satisfies:
\begin{equation}
  \label{eq:stability-condition-xc}
  x_c^k + \frac{\dot{x}_c^k}{\omega} = \omega \int_{t_k}^\infty 
      e^{-\omega(\tau-t_k)}x_z(\tau)d\tau
\end{equation}
then the solution of $x_c$ dynamics (TODO: eqref) remains bounded for all
$t$. Similar for $y_c$ dynamics. Regarding $z_c$:
\begin{equation}
  \label{eq:stability-condition-zc}
  z_c^k + \frac{\dot{z}_c^k}{\omega} = \frac{g}{\omega^2} +
      \omega \int_{t_k}^\infty e^{\omega(\tau-t_k)}z_z(\tau)d\tau
\end{equation}

MPC formulation, stability condition \eqref{eq:stability-condition-xc} can be 
enforced by writing it as a constraint on the control variables
$\dot{x}_z^{k+i}$:
\begin{equation}
  \label{eq:stability-constraint-xdot}
  \frac{1}{\omega}\frac{1-e^{-\delta\omega}}{1-e^{-N\delta\omega}}
    \sum_{i=0}^{N-1} e^{-i\delta\omega} \dot{x}_z^{k+i} =
    x_c^k + \frac{\dot{x}_c^k}{\omega} - x_z^k
\end{equation}
(using integral eqref{eq:stability-condition-xc} + piecewise linear ZMP trajectory
\eqref{eq:piecewise-linear-zmp-trajectory} + infinite replication). TODO:
infinite replication proof here.

Stability constraint on $z_c$ derived from \eqref{eq:stability-condition-zc}:
\begin{equation}
  \label{eq:stability-constraint-zdot}
  \frac{1-e^{-\delta\omega}}{1-e^{-N\delta\omega}}
    \sum_{i=0}^{N-1} e^{-i\delta\omega} \dot{z}_z^{k+i} =
    z_c^k + \frac{\dot{z}_c^k}{\omega} - z_z^k - \frac{g}{\omega^2}
\end{equation}
where $\dot{z}_z$ is set to zero beyond prediction horizon (TODO: proof here).

4. MPC Algorithm

MPC algorithm which solves QP problem at each iteration.

4.1, Formulation of the QP problem

Defining the decision variables vectors:
\begin{align}
  \dot{X}_\text{z}^k&=(\dot{x}_\text{z}^k, \dots, \dot{x}_\text{z}^{k+N-1})^T \\
  \dot{Y}_\text{z}^k&=(\dot{y}_\text{z}^k, \dots, \dot{y}_\text{z}^{k+N-1})^T \\
  \dot{Z}_\text{z}^k&=(\dot{z}_\text{z}^k, \dots, \dot{z}_\text{z}^{k+N-1})^T
\end{align}
QP problem:
\begin{align*}
  \min_{\dot{X}_\text{z}^k, \dot{Y}_\text{z}^k, \dot{Z}_\text{z}^k}
      &\sum_{i=1}^N
      \biggl[
          (\dot{x}_z^{k+i})^2 +
          (\dot{y}_z^{k+i})^2 +
          (\dot{z}_z^{k+i})^2 + \\
          &\beta \biggl(
              (x_z^{k+i} - x_f^{k+i})^2 +
              (y_z^{k+i} - y_f^{k+i})^2 +
              (z_z^{k+i} - z_f^{k+i})^2
          \biggr)
      \biggr]\\
      \textrm{s.t. } &\textrm{ZMP constraint \eqref{eq:zmp-constraint-3d}} \\
      &\textrm{stability constraints \eqref{eq:stability-constraint-xdot},
          \eqref{eq:stability-constraint-zdot}}
\end{align*}
where the cost function includes the decision variables for regularization 
purposes and a term which attempts to bring the ZMP to patch level whenever 
possible (TODO: actually to center of the footstep).

4.2. Algorithm

Todo.

\section{BHuman implementation}
BHuman + NAO.

\section{Conclusion}
Conclude by summarizing importance of the model + simulations in
separate ch.

