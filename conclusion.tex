\chapter{Conclusion}
\label{ch:conclusion}
\section{Results}
This thesis presents an architecture that integrates mapping, planning and 
control to generate humanoid gaits in \textit{World of Stairs} unknown 
environments. The use of \texttt{elevation\_mapping}
\cite{Fankhauser2018ProbabilisticTerrainMapping} together with a depth
sensor allows the robot to represent the surrouding environment as a map that
can be used by a RRT-based footstep planning module \cite{ECC19}
to generate a sequence
of footsteps. The Variable-Height CoM IS-MPC \cite{SYROCO18} 
has been implemented on NAO
humanoid robot upon the BHuman framework and the whole architecture has been
tested on multiple scenarios.

\section{Future Works}
The current architecture does not include a localization module, limiting the
potentialities of the robot. Localizing the robot inside the environment would,
in fact, provide a precise configuration of the humanoid, enabling 
\texttt{elevation\_mapping} to continuously build the map during locomotion.
A possible extension could be to develop such module using a Kalman filter
\cite{Bloesch2013StateEstimationLeggedRobots} or a factor graph
\cite{Wisth2019PreintegratedVelocityBiasEstimation}.
Another possible extension of this work could be that of developing a replanning
phase \cite{Griffin2019FootstepPlanningRoughTerrain}, allowing the robot to
work in dynamic environments. The combination of these two could give even
more autonomy to humanoid robots, further advancing current technology and
allowing their introduction into our society.

